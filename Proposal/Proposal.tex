\documentclass[12pt]{article}

\usepackage{amsmath}
\usepackage[margin = 1in]{geometry}
\usepackage{booktabs}
\usepackage{natbib}

% highlighting hyper links
\usepackage[colorlinks=true, citecolor=blue]{hyperref}


%% meta data

\title{Proposal: Analyzing Mens Tennis Grand Slam Performance}
\author{Kaitlyn Shevlin\\
  Department of Statistics\\
  University of Connecticut
}

\begin{document}
\maketitle


\paragraph{Introduction}
There are four Grand Slam tournaments in tennis: the Australian Open,
the French Open, Wimbledon, and the US Open, spanning from January to 
September \citet{Ifttennis2021Grand}. These notourious tournaments 
feature the world's best tennis players and are each played over a 
course of two weeks.  As with every sport, there are a variety of 
controllable, regulated factors as well as uncontrollable, extraneous 
factors that affect the event. For example, one of these is the type of 
court that the matches are played on, which vary at each tournament. 
This paper will evaluate Mens Grand Slam tennis championship matches. 
I have chosen this topic because tennis is an interest of mine and I am 
curious to further explore the different variables that impact player 
performance.

\paragraph{Specific Aims}
Using this data set, I will conduct exploratory analysis to determine
which factors, if any, are useful in predicting the winner of a Grand
Slam championship match. This includes examining if ATP ranking is 
indicative of who will win the match. In addition, I will conduct 
further research in attempt to determine if certain multi-title holders 
perform better at different tournaments due to the surface, as well as 
if being right handed or left handed aids performance at certain 
tournaments. There is the idea of home court advantage in all of sports, 
so it will be interesting to examine if that holds true in tennis Grand 
Slams, as well as further delving into the notion that Spanish players 
perform better at the French Open- which may be explained by Spanish 
players grow up practicing on clay courts \citet{Jurejko2018French}. 
The objective of the analysis is to provide a better understanding of 
how certain factors affect players as well as potential predictive 
modeling.
 
\paragraph{Data}

There are 289 observations with 4 data points per year, for each major 
Grand Slam with the exception of Wimbledon in 2020 as it was cancelled 
due to COVID-19. The dataset includes YEAR: year tournament started; 
TOURNAMENT: tournament type; WINNER: winner's name; RUNNER UP: runner-up's 
name; WINNER ATP RANKING: winner's ATP ranking, RUNNER UP ATP RANKING: 
runner-up's ATP Ranking; WINNER LEFT OR RIGHT HANDED: whether the winner 
is right or left handed; TOURNAMENT SURFACE: surface tournament is played 
on; and WINNER PRIZE: prize money won at each tournament in local 
denominations. For the purpose of this research, winner prize money will 
not be considered. ATP stands for Association of Tennis Professionals and 
the ATP ranking is a player's ranking based on total points earned in 19 
official ATP-certified men's events. It is capped at 19 events so if a 
player goes over, their best 19 results are counted \citet{Nag2022Tennis}. 
In this study, the ATP ranking is the player's ranking going into the 
tournament.

\paragraph{Research Design and Methods}

I will look at a mixed-effect models which will allow for looking at a 
mass of information with different players to estimate the effects of 
common variables. Players appear in multiple Grand Slams throughout the 
years so they appear multiple times in the data. I will also hone in on 
individualized player models to see how specific factors affect them and 
look for trends among their matches.

\paragraph{Discussion}
I am expecting to find that factors such as ATP ranking and court type 
have some impact on who the winner is. I would expect that players are 
used to one court type over another and that has some impact on their 
performance. In addition, typically players of a higher rank have earned 
that title for a reason and are favored to win. Players' dominant hand 
also has an impact as it affects the spin and motion on the ball \citet{Jurejko2018French}. 
The potential impacts of my work could help to better predict the winners 
of the Grand Slam tournaments. If the investigation is not what I expect, 
then it will most likely appear that there is no correlation between any 
of these factors which would be puzzling and require further analysis. 
Since I am only looking at a few factors, there are bound to be many more 
which have an effect on this data, or some hidden variables that must be 
explored.

\paragraph{Conclusion}
This research project will provide insight into what elements affect players 
or a specific player in the four major tennis Grand Slams. Although analysis 
has been done on various aspects of this data, my research will look at data 
from a wider range of dates as well as a different combination of factors. 
The aim is to corroborate previous research done on this project and better 
understand how different factors affect different players and overall 
observations. 

\bibliography{citations.bib}
\bibliographystyle{chicago}

\end{document}